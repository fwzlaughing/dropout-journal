\begin{abstract}
Massive open online courses (MOOCs) have developed rapidly in recent years, and have attracted millions of online users.
However, a central challenge is the extremely high dropout rate --- recent reports show that the completion rate in MOOCs is below 5\%~\cite{onah2014dropout,Kizilcec:2013:DDA:2460296.2460330,Seaton2014Who}. What are the major factors that cause the users to drop out? What are the major motivations for the users to study in MOOCs? In this paper, employing a  dataset from XuetangX\footnote{https://xuetangx.com}, one of the largest MOOCs in China, we conduct a systematical study for the dropout problem in MOOCs. 
We found that the users' learning behavior can be clustered into several distinct categories.
Our statistics also reveal high \textit{correlation} between dropouts of different courses and strong \textit{influence} between friends' dropout behaviors. Based on the gained insights, we propose a Context-aware Feature Interaction Network (CFIN) to model and to predict users' dropout behavior. CFIN utilizes context-smoothing technique to smooth feature values with different context, and use attention mechanism to combine user and course information into the modeling framework. Experiments on two large datasets show that the proposed method achieves better performance than several state-of-the-art methods. The proposed method model has been deployed on a real system to help improve user retention.
%\modelname{} utilizes a feature context-smoothing technique to deal with the sparsity problem, and use the attention mechanisms to combine user and course-specific information into the modeling framework.
%, by incorporating context information in MOOCs. The proposed method achieves state-of-art performance on both KDDCUP open dataset and XuetangX large-scale dataset. Moreover, the framework has been deployed on XuetangX to help users retention.


%Massive open online courses (MOOCs) have developed rapidly in recent years, and have attracted millions of users to receive education online. However, a central challenge is the high dropout rate  for MOOCs learners. Therefore, it is important to understand why users drop out and to predict which users would drop out.

%This paper aims to tackle these two questions. First, we conduct in-depth analyses to identify factors which influence dropout, and propose an effective cluster method to understand users' different engagement pattern. Then we design a series of novel methods to learn behavior features from log data and to learn individual preference based on user-course enrollment graph. Finally, we adopt a unified co-training framework to combine all features. This algorithm allows each model to learn from each other by making use of unlabeled data, which enhances the performance of personalized prediction. The proposed method achieves a state-of-art performance on KDDCUP open dataset, and the system has been deployed on XuetangX\footnote{https://xuetangx.com}  platform.


	

 \end{abstract}